\documentclass{article}

\usepackage{a4wide}
\title{Some Guidelines to Writing a Research Proposal and a Report}
\author{Nicky van Foreest}
\begin{document}
\maketitle

\section{Introduction}
\label{sec:introduction-1}

Over the years I have noticed that my thesis students have the same
type of questions about starting and organizing their reports. This
document contains an overview of the answers I have been giving time
and again. So, what should you do with this document? 
\begin{enumerate}
\item Read it.
\item Check your writing against the questions and checklists below.
\item If your report does not answer one of these questions or fails
  to meet a point on the checklist, then do it.
\item If you do more than what is here, it is probably superfluous, so
  scrap it.
\end{enumerate}

\section{The Most Important Rule}
\label{sec:most-important-rule}

Think about how you read an article in the papers. If you are honest,
you'll admit that as soon as you get bored, you shove it aside and you
start reading/doing something else. Be very aware that your reader
will do the same with your work if it gets
boring/unclear/unmotivated/too abstract.

Thus, in general a reader has a number questions when s/he reads your
work:
  \begin{enumerate}
  \item Why should i be interested in your work? (Typically, I am not,
    and I feel like I am wasting my time.) \label{item:1}
  \item Why should I believe you or any of claims you make? (Typically, I don't.) \label{item:2}
  \item Why should I (try to) understand what you write?  (Typically,
    I have no clue what you are trying to achieve.) \label{item:3}
  \end{enumerate}

Hence: 
\begin{center}
RULE ONE: HAVE THE ABOVE THREE QUESTIONS IN THE BACK OF YOUR MIND, ALWAYS, WHEN YOU WRITE.
\end{center}

Perhaps a few comments are in order here. 
\begin{enumerate}
\item Realize that a reader is not principally interested in what you
  write.  A reader (i.e., you as a reader, not as a writer) hates to
  read irrelevant, unrelated stuff. By all means, remove superfluous
  material in your thesis, it doesn't add anything, on the contrary,
  it confuses and \emph{bores} the reader. About anything you write,
  think about if you need it; in case of the slightest doubt, scrap
  it.
\item Realize that you can write down any claim you like, e.g., `There
  are dogs that fly backwards and read the papers at the same time.'
  Of course, I don't believe this claim, but you can write it down
  nonetheless. Similarly, with respect to making research claims, you
  can write anything you like, but this the fact that you can do this
  does not make it true. Thus, to make people believe your claims, you
  should explain \emph{explicitly} why (and how and the extent to
  which) your methods provide any credibility to any claim you make.
\item Finally, to me as a reader your topic is new. I don't have all
  the relations at the back of my mind to understand your
  reasonings. Hence, make relations between different topics
  explicit. Explain what you do/write about, and why you do/write about
  this. Otherwise I am left clueless about your intentions/goals.
\end{enumerate}

So, for each section/paragrapy check that you did not sin against RULE
ONE.


\section{General Guidelines}
\label{sec:most-important-rule}


\begin{itemize}
\item The thesis should be a piece of work in which each of the taken
  steps are defined and motivated. Since, typically, the problem under
  investigation is too hard to solve directly, the only way to obtain
  confidence in the final answer is to show that you asked a sensible
  question and you followed a sensible method to obtain the answer. 
\item By the above, write `question driven'.  
\begin{enumerate}
\item Start with making a list of relevant questions.  These questions
  should be `natural', that is, they should make sense for a reader
  without detailed background.
\item Write the questions down!
\item Order these questions in a logical way, such that once the
  questions are answered, your work is complete.
\item Answer the questions.
\item Once you think you are ready: 
  \begin{itemize}
  \item Have you covered all questions? If not, repair.
  \item Have you covered more than these questions? Why? If you cannot
    motivate these extra points/topics, remove them. 
  \item Check that you do not mix answers to different questions.
  \end{itemize}
\item Check your adherence to RULE ONE.
\item Once you completed the steps  above, stop. 
\item At each step: use your common sense to determine what to
  do. 
\item Start with simple statements/models, and expand, rather than
  start with something complicated/general and end up with nothing.
\end{enumerate}
\end{itemize}



\section{Report}
\label{sec:report}



\subsection{Introduction}
\label{sec:introduction}

Here you should provide the answer to Question 1 of Rule one. 

\begin{enumerate}
\item Provide context.
  \begin{enumerate}
\item Where does the problem come from? (Concrete example, practical context, initial motive.)
\item What is the problem couched in more general terms?
\item Why is this problem relevant?
  \end{enumerate}
\item What did others do? (Provide theoretical context, investigate,
  for the problem you'll address.  This is about getting the right
  problem.)
  \begin{enumerate}
  \item To what similar problems does the general problem relate, what
    are overlaps and differences?  (Here you just mention this point
    briefly, just one or two most relevant references.  In the
    literature section you should address this question thoroughly. )
  \item What methods have been developed by others for these problems?
    To what extent can you use these methods, to what extent not?
    (Again, brief discussion, main methods.  In the literature
    section, expand on this point.)
  \end{enumerate}
\item Why is new work necessary?
  \begin{enumerate}
  \item What makes your (general) problem unique? (Why should the
    reader care to know it?)
\item Why is a new method necessary, or why are the
  existing/known/published methods not sufficient? 
  \end{enumerate}
\item What is your contribution? What do you have to bring to the
  table? (This is about getting the problem right.)
  \begin{enumerate}
  \item What problem (research question) do you precisely address in
    the paper? (If you have more than one main topic, typically remove
    all topics, until just one remains.) Think about the type of
    answer you can give to your research question. For example, if the
    question is this: `How to minimize the cost of inventory?', the
    obvious answer is not to have inventory at all. Most surely, this
    answer is not what you have in mind. The point is that the initial
    quesion was silly to begin with. A better question is: `How
    increase the revenues with my inventory system by 5\%?'
  \item What method(s) are you going to use to get the answer? (Motivate methodology.)
  \item What main result/insight do you achieve? (Provide a hint what
    the reader can expect. The main results will of course be
    discussed in the conclusion section.) Why is your work useful?
  \end{enumerate}
\item Structure/overview of the paper.
\end{enumerate}



\subsection{Literature}
\label{sec:methods-section}

Basically, I want to see here that you did your homework. 

\begin{itemize}
\item What are the overlaps and differences between the problems in
  the literature and your problem? Discuss the details, show that you
  understand what you are talking about. 
\item What methods have others (papers) used to tackle these similar
  problems? 
\item To what extent are these known methods useful, can be applied to
  your problem? If so, how, if not why not? Here again, show that you
  know what you are talking about. (Do not discuss methods you will
  not use. Recall RULE ONE: The reader is not interested.)
\item How can you characterize your problem and characterize the
  methods so that you can make a match between method and problem?
\item How do all the discussed concepts relate to each other and to your problem in particular?
\end{itemize}

Besides providing context, literature can  be used to
\begin{itemize}
\item find (inspiration for) the design
\item find other models/designs. (Compare these other models/designs with the chosen design.)
\item find performance measures to assess the new design
\item find methods to validate the design
\end{itemize}


\subsection{Methods section}
\label{sec:methods-section}

Basically, here you have to convince readers that the methods you are
going use lead to correct/plausible answers. Recall Question 2 of Rule one.

\begin{itemize}
\item How are you going to approach your problem?
\item Do you have sub-questions? They help to decompose the main
  problem, hence organize your work. When these sub-questions are
  answered, there should be a direct way to answer the your research
  question. 
\item What would be suitable methods to solve your problem? why? Use
  your common sense to determine suitable methods.
\item What are the (dis)advantages of these methods, in the light of
  your problem?
\item How are you going to decide which method(s) is the most useful for your problem? Why?
\item Finally, which method are you going to use (e.g., simulation)?
  Why?
\end{itemize}

If possible, try to relate all the above to a conceptual model. Recall
Question 3 of Rule one. The reader has no mental framework of how you
are approaching the problem. A figure can help to organize your
reasoning and help with Question 3 of Rule one.

\begin{itemize}
\item What is the `thing' you want to control, i.e., what is the unit of analysis?
\item What are the `things' you can control? 
\item Which KPIs do you use to  evaluate the effect/success of these controls?
\item Is the list of KPI's complete, consistent? (Are they defined properly?)
\item How does the conceptual model relate to your problem? 
\item How do the sub-questions fit into the conceptual model?
\item How does your conceptual model differ/overlap with with existing
  models?
\item If possible, use your model to formulate a hypothesis, e.g., `I
  suspect that by controlling such and so, the effect will be like
  this?'
\item If possible, quantify the hypothesis/relation, e.g., by
  doubling the \ldots, the profit increases by\ldots.
\end{itemize}


\subsection{Experiments}
\label{sec:experiments}


\begin{itemize}
\item Describe a production process in the sequence in which the
  products are produced. Include only the relevant parts, parts that
  relate to the research question.
\item What scenarios are you going to use? Why these? How do you
  collect the data? Why is the data ok?
\item How will you test/ensure the correctness of your work, i.e.,
  validation.? How to gauge the results/claims? How to check the
  reliability of the obtained data/insights?
\item Include only figures and tables that can be related to the
  research question.
\item Try to make the figures and tables self-explanatory. Describe in
  the captions what the table/figure is about, and what can be
  observed. For figures, state what is on the $x$ and $y$ axis, e.g.,
  `the profit as a function of time'.
\item Check that figures are coherent, include the same type of data.
\item Make observations explicit.  Move conclusions to the conclusion section
\item If you qualify results as `good' or `bad', provide benchmarks to
  show why the result is `bad' or `good'.
\item How does all this work relate to the main question? Why and how
  does the chosen method and data will lead to an answer of the
  research question?
\item Does your report contain sufficient detail and organization that
  somebody else can repeat your work? If not, repair this.
\end{itemize}


\subsection{Results}
\label{sec:results}

\begin{itemize}
\item What general insights have  you obtained?
\item What are the limitations?
\item To what extent do they apply to your initial/practical problem?
\item How robust are your results? You might carry out sensitivity
  analysis as convincing evidence that your results make sense in
  practical settings. Sensitivity analysis adds a lot to the
  credibility of your work.
\end{itemize}


\subsection{Pruning}
\label{sec:pruning}

One of the last steps of writing your thesis is to prune it. Use the
checklist below to see whether you did this.

\begin{itemize}
\item Write down, in about five bullets, the main structure of your
  thesis. Do not exceed 100 words. Check your table of contents agains
  these five bullets.
\item For each paragraph, think about the goal of this piece of
  writing. What do you want to achieve here? In view of your main
  problem, can it be left out? In case of the slightest doubt, remove
  it.
\item Check the tense of the verbs. Wherever possible, use present
  tense. Avoid passive voice if possible (passive voice becomes boring
  and long-winded pretty soon.)
\item For each sentence you write, think about whether it is strictly
  necessary to include. Does it help/support your main topic/message?
  If not, remove
\item For all remaining sentences, think about its proper place.
\item Your thesis should be self-contained.
\item The writing should be coherent and logical.
\item At each chapter/section, announce briefly your intent.  Check
  that you do not embark on difficult explanations before giving an
  hint about what you intend to achieve. 
\end{itemize}

\section{Reading Sessions with Others}
\label{sec:read-sess-with}

If you read a report of somebody else, answer the questions below,
that is,  write down your answers to all these questions!
\begin{itemize}
\item Before you start reading, formulate explicitly your initial
  opinion about the report.
  \begin{itemize}
  \item What do you think about the size? Too thick most probably.
  \item What do you think about the title?
  \item Would you actually read this document from cover to cover? If
    so, why? If not, why not?
  \end{itemize}
\item Read the table of contents. 
  \begin{itemize}
  \item What do you think about it?
  \item Is it clear/unclear? Why?
  \item Do you get an impression about the contents of the report?
  \item Do you understand the organization?
  \end{itemize}
\item Start at the first sentence of the introduction
  \begin{itemize}
  \item Stop reading as soon as you feel you like to stop reading. Try
    to understand why you want to stop. Write your answer down!
  \item What do you expect to see in an introduction? Are your
    expectations met? If this is not the case, what is missing? 
  \item Move to some arbitrary position in the introduction,  start
    reading, and stop immediately at the slightest feeling that you
    dislike to continue. Why do you stop? What expectations of you, as
    a reader, are not met?
  \end{itemize}
\item Communicate your dissatisfaction to the author.
\item The author has to repair your problem \emph{on paper, not verbally}.
\item And so on, i.e., Apply the above reading style to Chapter
  2. Stop as soon as you dislike to continue, and figure out what
  makes you want to stop reading. What is missing, what is wrong with
  the text, why?
\end{itemize}

Apply all assembled criticism to your own thesis, and improve your
work accordingly. This will be very rewarding:
\begin{itemize}
\item It will improve your writing skills
\item It will save a \textit{lot} of time, as you need to write less
  to get the main message across.
\item It will increase the quality of the feedback of your
  supervisors, as they better understand your work.
\end{itemize}

\section{Presentation}
\label{sec:old-stuff}

I noticed that presentations are often not as clear and explicit on
certain points as they should have been. Not being explicit about the
aims is a problem as it obscures the results and confuses, hence
bores, the audience. 

Please check that you \textit{explicitly} deal with \textit{each} of
the points below in your presentation.  You can either mention these
points verbally during the presentation, or include it in one or two
lines of the slides, or otherwise, but make sure that the answers to
these points come across.

\begin{itemize}
\item What is your target/intended audience (e.g., researchers,
  managers, students, \ldots). Who needs this work? Why?
\item What is the problem, its context, scope, and relevance?  Provide
  motivation. Why is it a problem? Who said so? How do you know? Mind,
  if the audience does not understand your problem, does not care
  about your problem, you can just as well skip the rest. Hence, this
  step is essential, the most important part of the talk.
\item What is the main question? What are the sub-questions? 
\item What is the product you deliver?  Why do you carry out this
  research? What is the aim? Why do you want to know this? What is the
  benefit/off-spin?
\item What do you want to with your insights, results, product? Or otherwise,
  what should your audience do with it?
\item What results do you expect to obtain? Is there a hypothesis you
  can formulate, try to refute or accept?
\item Explain the conceptual model. What relation do you want to show?
\item What data do you need to support your hypotheses or to show the
  relations/correlations in the conceptual model?
\item What methods, e.g., survey, questionnaire, do you use to obtain
  the insights/results? Why that method?
\item What is the relation between the method and the main question(s)
  you asked? Show how the (expected) results answer your question,
  that is, make the relation explicit.
\end{itemize}


\end{document}

%%% Local Variables:
%%% mode: latex
%%% TeX-master: t
%%% End:
